\documentclass[twoside]{article}

\usepackage[sc]{mathpazo} % Use the Palatino font
\usepackage[T1]{fontenc} % Use 8-bit encoding that has 256 glyphs
\linespread{1.05} % Line spacing - Palatino needs more space between lines
\usepackage{microtype} % Slightly tweak font spacing for aesthetics

\usepackage[hmarginratio=1:1,top=32mm,columnsep=20pt]{geometry} % Document margins
\usepackage{multicol} % Used for the two-column layout of the document
\usepackage[hang, small,labelfont=bf,up,textfont=it,up]{caption} % Custom captions under/above floats in tables or figures
\usepackage{booktabs} % Horizontal rules in tables
\usepackage{float} % Required for tables and figures in the multi-column environment - they need to be placed in specific locations with the [H] (e.g. \begin{table}[H])
\usepackage{hyperref} % For hyperlinks in the PDF

\usepackage{lettrine} % The lettrine is the first enlarged letter at the beginning of the text
\usepackage{paralist} % Used for the compactitem environment which makes bullet points with less space between them

\usepackage{abstract} % Allows abstract customization
\renewcommand{\abstractnamefont}{\normalfont\bfseries} % Set the "Abstract" text to bold
\renewcommand{\abstracttextfont}{\normalfont\small\itshape} % Set the abstract itself to small italic text

\usepackage{titlesec} % Allows customization of titles
\renewcommand\thesection{\Roman{section}} % Roman numerals for the sections
\renewcommand\thesubsection{\Roman{subsection}} % Roman numerals for subsections
\titleformat{\section}[block]{\large\scshape\centering}{\thesection.}{1em}{} % Change the look of the section titles
\titleformat{\subsection}[block]{\large}{\thesubsection.}{1em}{} % Change the look of the section titles

\usepackage{fancyhdr} % Headers and footers
\pagestyle{fancy} % All pages have headers and footers
\fancyhead{} % Blank out the default header
\fancyfoot{} % Blank out the default footer
\fancyhead[C]{EECS 584 $\bullet$ October 2nd, 2014 $\bullet$ Barzan Mozafari} % Custom header text
\fancyfoot[RO,LE]{\thepage} % Custom footer text

%----------------------------------------------------------------------------------------
%	TITLE SECTION
%----------------------------------------------------------------------------------------

\title{\vspace{-15mm}\fontsize{24pt}{10pt}\selectfont\textbf{Extending BlinkDB to Support Linear Matrix Operators}} % Article title

\author{
\large
\textsc{Nathan Harada, Julian Katz-Samuels}\\[2mm] % Your name
\normalsize University of Michigan \\ % Your institution
\normalsize $\{$nharada, jkatzsam$\}$@umich.edu
\vspace{-5mm}
}
\date{}

%----------------------------------------------------------------------------------------

\begin{document}

\maketitle % Insert title

\thispagestyle{fancy} % All pages have headers and footers

%----------------------------------------------------------------------------------------
%	ABSTRACT
%----------------------------------------------------------------------------------------

\begin{abstract}

We propose to extend BlinkDB to support linear matrix operators on large data sets. These operations will provide estimated error and confidence intervals similar to the current BlinkDB functions. We will develop the algorithms and commands, implement them in the current BlinkDB code, analyze their performance, and eventually merge our changes back into the open source project.

\end{abstract}

%----------------------------------------------------------------------------------------
%	ARTICLE CONTENTS
%----------------------------------------------------------------------------------------

\begin{multicols}{2} % Two-column layout throughout the main article text

\section{Introduction}

\lettrine[nindent=0em,lines=3]{A}s modern OTLP databases continue to grow in size, traditional database queries are beginning to take an unfeasably long time to complete. Sampling databases, such as BlinkDB, attempt to offer meaningful results in a short period of time by providing an estimate of a query, along with a measure of the expected error. For staticians and data scientists, this "close enough"  answer may be within acceptable error limits for data analytics and business intelligence. However, BlinkDB is currently limited to a few basic statistical operators. We propose extending BlinkDB's functionality to include linear matrix operators. Matrix operations are essential for many statistical analysis techniques, such as regression or curve fitting. By providing a "good enough" analysis of the data, we hope to reduce computing time required for such queries while maintaining meaningful accuracy. 

Our proposal is three fold. First we plan to develop algorithms to calculate meaningful error from linear matrix operators. Second, we plan to implement these operators in BlinkDB, as well as extending the BlinkDB query language to support these operations. Last, we will test our results on both standard benchmarks and real world data, evaluating both performance and accuracy on systems of various sizes.

%------------------------------------------------

\section{Matrix Operators}
\subsection{Proposed Work}
\subsection{Potential Roadblocks}

\section{BlinkDB Extension}
To add matrix operators to BlinkDB, we will need to both implement the underlying mathematical algorithms, as well as change the query language to support these operations. 
\subsection{Proposed Work}
We propose introducing functions to both transform a selection of data into a matrix, as well as run matrix operations on this data. The current SQL specification implemented in BlinkDB does not offer any way to specify matrix operations, thus we will have to create new scema to do so. There are a few possiblities to consider. For our purposes we will assume that we require all matricies to be stored in a the database and we will not accept them at query-time. Our first possibility is to nest functions and provide a function that creates matricies, for example, to calculate the total value of an inventory:

\begin{verbatim}
SELECT DOTPROD(VEC(prices), VEC(counts))
FROM inventory
WHERE prices > 10
\end{verbatim}

\subsection{Potential Roadblocks}

\section{Testing}
\subsection{Proposed Work}
\subsection{Potential Roadblocks}

\section{Conclusion}

%----------------------------------------------------------------------------------------
%	REFERENCE LIST
%----------------------------------------------------------------------------------------

\begin{thebibliography}{99} % Bibliography - this is intentionally simple in this template

\bibitem[Figueredo and Wolf, 2009]{Figueredo:2009dg}
Figueredo, A.~J. and Wolf, P. S.~A. (2009).
\newblock Assortative pairing and life history strategy - a cross-cultural
  study.
\newblock {\em Human Nature}, 20:317--330.
 
\end{thebibliography}

%----------------------------------------------------------------------------------------

\end{multicols}

\end{document}